\documentclass[10pt,a4paper]{article}
\usepackage[utf8]{inputenc}
\usepackage[francais]{babel}
\usepackage[T1]{fontenc}
\usepackage{amsmath}
\usepackage{amsfonts}
\usepackage{amssymb}
\usepackage{geometry}
%\geometry{hmargin=3.5cm,vmargin=2cm}
\author{Félix Piédallu \& Nicolas Lebbe \\ {\footnotesize Document édité avec \LaTeX}}
\title{Le théorème de Baire : Énoncé et démonstration}
\date{}
\begin{document}
\thispagestyle{empty}
\part*{Énoncé du théorème de Baire}
Soit E un espace vectoriel normé et complet (Espace de Banach).
\begin{center}
\begin{tabular}{|l|}
\hline
Soit $(O_n)_n$ une suite d'ouverts denses de E.\\
Alors $\bigcap_{n \in\mathbb{N}} O_n$ est dense dans E (Mais non ouvert). \\
\hline
\multicolumn{1}{c}{$\Leftrightarrow$} \\
\hline
Si $(F_n)_n$ est une suite de fermés d'intérieur vide,\\
$\bigcup_{n \in \mathbb{N}} F_n$ est un ensemble d'intérieur vide. \\
\hline
\end{tabular} \\
\end{center}
En effet, $A$ est d'intérieur vide $\Leftrightarrow$ son complémentaire est dense dans E. 
\part*{Preuve}
Soit $x \in E$ et $r >0 $. Montrons que 
\begin{equation}
B=B(x,r) \cap (\bigcap_{n \in \mathbb{N}} O_n) \neq \emptyset
\end{equation}
(En effet, A dense $\Leftrightarrow \forall x, \forall r, B(x,r) \cap A \neq \emptyset $)\\

$O_1$ dense $\Rightarrow \exists (x_1, r_1)$ tels que $B_1 = B(x_1,r_1) \subset (O_1 \cap B)$ et $r_1 < {r \over 3}$.\\
Par récurrence, on définit :
\begin{equation*} B_i = B(x_i, r_i) \subset (O_i \cap B_{i-1}) \text{ et } r_i < {r_{i-1} \over 3} 
\end{equation*}

On obtient donc que pour tout $i, ||x_{i+1}-x_{i}|| \leq r_{i+1} \leq {r \over 3^i}$.\\

Donc $ \sum x_{n+1} - x_n$ converge absolument.\\
Enfin, par "critère suite-série" dans un espace de Banach, la suite des $x_n$ converge vers un élément $x_\infty \in E$.\\

De plus, $B_n$ étant une suite de boules emboîtées, elle vérifie :\\
$\forall p \in \mathbb{N} : \\ \forall n \geq p, B(x_n, r_n) \subset B(x_p, r_p) \subset O_p \cap … \cap O_1 $. \\
D'où $x_\infty \in O_p \cap … \cap O_1$.

Comme la relation est vraie pour tout $p$, on a $x_\infty \in \bigcap_{n \in \mathbb{N}} O_n$. Et alors :\\
\begin{equation*}
\left\{\begin{array}{l}
x_\infty \in B(x, r) \\
x_\infty \in \bigcap_{n \in \mathbb{N}} O_n
\end{array}\right . \Longrightarrow 
x_\infty \in B(x,r) \cap (\bigcap_{n \in \mathbb{N}} O_n)
\end{equation*}

On a donc $(1)$ et ce qui achève la démonstration.
\end{document}